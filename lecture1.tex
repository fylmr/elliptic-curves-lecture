\documentclass[12pt]{article}
\usepackage{amsmath,amssymb,amsthm}
\usepackage{algorithm}
\usepackage[noend]{algpseudocode} 
\usepackage{enumitem}

%---enable russian----

\usepackage[utf8]{inputenc}
\usepackage[russian]{babel}


%---tikz----
\usepackage{tikz}
\usetikzlibrary{arrows, chains, matrix, positioning, scopes, patterns, shapes}
\usepackage{pgfplots, subfigure}

\usepackage{biblatex}

% MACROS 
% PROBABILITY SYMBOLS
\newcommand*\PROB\Pr 
\DeclareMathOperator*{\EXPECT}{\mathbb{E}}


% GROUPS/DISTRIBUTIONS/SETS/LISTS
\newcommand{\N}{{{\mathbb N}}}
\newcommand{\Z}{{{\mathbb Z}}}
\newcommand{\Q}{{{\mathbb Q}}}
\newcommand{\R}{{{\mathbb R}}}
\newcommand{\F}{{{\mathbb F}}}
\newcommand{\PP}{{{\mathbb P}}}
\newcommand*{\IZ}{\ensuremath{\mathbb{Z}}}
\newcommand*{\IN}{\ensuremath{\mathbb{N}}}
\newcommand*{\IR}{{{\mathbb R}}}
\newcommand{\Zp}{\ints_p} % Integers modulo p
\newcommand{\Zq}{\ints_q} % Integers modulo q
\newcommand{\Zn}{\ints_N} % Integers modulo N
\newcommand{\Zr}{\ensuremath{\mathbb{Z}/r\mathbb{Z}}} % Integers modulo N
\newcommand*{\dDR}{\mathrm{d}} %de-Rham-Differential (the d in dx, dy, dz and so on)
\newcommand{\transpose}{\mkern0.1mu^{\mathsf{t}}}
\newcommand*{\union}{\mathbin{\cup}}

% Landau 
\newcommand{\bigO}{\mathcal{O}}
\newcommand*{\OLandau}{\bigO}
\newcommand*{\WLandau}{\Omega}
\newcommand*{\xOLandau}{\widetilde{\OLandau}}
\newcommand*{\xWLandau}{\widetilde{\WLandau}}
\newcommand*{\TLandau}{\Theta}
\newcommand*{\xTLandau}{\widetilde{\TLandau}}
\newcommand{\smallo}{o} %technically, an omicron
\newcommand{\softO}{\widetilde{\bigO}}
\newcommand{\wLandau}{\omega}
\newcommand{\negl}{\mathrm{negl}} 

% Misc
\newcommand{\eps}{\varepsilon}
\newcommand{\inprod}[1]{\left\langle #1 \right\rangle}

 
\newcommand{\handout}[5]{
  \noindent
  \begin{center}
  \framebox{
    \vbox{
      \hbox to 5.78in { {\bf  } \hfill #2 }
      \vspace{4mm}
      \hbox to 5.78in { {\Large \hfill #5  \hfill} }
      \vspace{2mm}
      \hbox to 5.78in { {\em #3 \hfill #4} }
    }
  }
  \end{center}
  \vspace*{4mm}
}

\newcommand{\lecture}[4]{\handout{#1}{#2}{#3}{Оформил #4}{Лекция #1}}

\newtheorem{theorem}{Теорема}
\newtheorem{corollary}[theorem]{Следствие}
\newtheorem{lemma}[theorem]{Лемма}
\newtheorem{observation}[theorem]{Observation}
\newtheorem{proposition}[theorem]{Предложение}

\theoremstyle{definition}
\newtheorem{definition}[theorem]{Определение}

\newtheorem{claim}[theorem]{Утверждение}
\newtheorem{fact}[theorem]{Факт}
\newtheorem{assumption}[theorem]{Предположение}

\theoremstyle{definition}
\newtheorem{examples}[theorem]{Примеры}

\theoremstyle{definition}
\newtheorem{example}[theorem]{Пример}

% 1-inch margins
\topmargin 0pt
\advance \topmargin by -\headheight
\advance \topmargin by -\headsep
\textheight 8.9in
\oddsidemargin 0pt
\evensidemargin \oddsidemargin
\marginparwidth 0.5in
\textwidth 6.5in

\parindent 0in
\parskip 1.5ex

\addbibresource{books.bib}

\begin{document}
    
    \lecture{1 --- 06.09.2019}{Осень 2019}{Лектор: Елена Киршанова}{Филипп Максимов}

    \printbibliography

    \section{Введение}
    
        $\F_q$ — конечное поле, $|\F_q|=q=p^k$, $p$ — прсотое, $K$ — поле, $\Bar{K}$ — алгебраическое замыкание.
        
        \subsection{Определения}
        
            \begin{definition} Уравнение Вейерштрасса в проективных координатах 
            — уравнение степени 3 вида 
                \begin{equation}
                \label{eq:weierstrassF}
                    F: Y^2Z + a_1 X Y Z + a_3 Y Z^2 = X^3 + a_2 X^2 Z + a_4 X Z^2 + a_6 Z^3
                \end{equation}
                где $a_i \in K$.
                Уравнение Вейерштрасса \textbf{гладкое} (или несингулярное), если для любых проективных точек $P=(X:Y:Z) \in \PP^2(K)$ \footnotemark[1], удовлетворяющих условию (\ref{eq:weierstrassF}), хотя бы одна из частных производных $\frac{dF}{dX},\frac{dF}{dY},\frac{dF}{dZ}$ не обращается в $0$ на $P$. Если все три частных производные обращаются в $0$ хотя бы на одной точке $P$ (точке сингулярности), (\ref{eq:weierstrassF}) — сингулярное уравнение.
                \footnotetext[1]{проективная плоскость над $K$ — множество классов эквивалентности на $K^3\setminus \{0,0,0\}$, т.е. $\overrightarrow{X} \sim \overrightarrow{Y}$, если $x_1=u\cdot y_1,x_2=u \cdot y_2, x_3=u \cdot y_3$ }
            \end{definition}
            
            \begin{definition} 
                \textbf{Эллиптическая кривая} $E$ (алгебраическая кривая рода 1) — множество всех точек в $\PP^2(K)$, удовлетворяющих гладкой кривой (\ref{eq:weierstrassF}). 
                
                Существует всего одна точка в $E$ с координатой $Z=0\,$: $(0:1:0)$. Обозначаем эту точку $\bigO$, называем точкой в бесконечности. 
            \end{definition}
            
            \begin{definition} 
            \label{eq:weierstrassaffine}
                Уравнение Вейерштрасса в аффинных координатах $(x=X/Z, y=Y/Z)$:
                \begin{equation}\label{eq:weierstrassequation}
                    f: y^2+a_1xy + a_3y = x^3 + a_2x^2 + a_4x + a_6
                \end{equation}
                Тогда $F(K) = \{ (x,y) \in K \times K: f(x,y)=0 \} \union \{\bigO\}$.
                
                Если $a_i \in K \  \forall i$, то будем говорить, что кривая $E$ определена над $K$.
            \end{definition}
            
            \begin{definition}
                Обозначим
                \begin{align}
                    d_2 &= a_1^2 + 4a_2 \\ \nonumber
                    d_4 &= 2a_4 + a_1a_3 \\ \nonumber
                    d_6 &= a_3^2 + 4a_6 \\ \nonumber
                    d_8 &= a_1^2a_6 + 4a_2a_6 - a_1a_3a_4 + a_2a_3^2 - a_4^2 \\ \nonumber
                    c_4 &= d_2^2 - 24d_4 \\ \nonumber
                    \text{Для проверки: } 4d_8 &= d_2d_6 - d_4^2
                \end{align}
                Тогда \textbf{дискриминант} уравнения (\ref{eq:weierstrassaffine}) определяется как 
                \[
                    \Delta = -d_2^2d_8 - 8d_4^3-27d_6^2+9d_2d_4d_6,
                \]
                А j-инвариант эллиптической кривой $E$, $j(E)$, определяется как 
                \[
                    j(E) = \frac{c_4^2}{\Delta}
                \]
            \end{definition}
            
            \begin{theorem}[]
                \autocite[Thm. 1.4]{silverman}
                Кривая, заданная уравнением Вейерштрасса, может быть классифицирована как:
                \begin{enumerate}[itemsep=0pt, topsep=0pt, partopsep=0pt]
                    \item Несингулярная $\iff \Delta \neq 0$ ($\implies$ задаёт эллиптическую кривую) 
                    \item Кривая, обладающая узлом (node) $\iff \Delta = 0, c_4 \neq 0$ 
                    \item Кривая, обладающая точкой перегиба (cusp) $\iff \Delta = c_4 = 0$
                \end{enumerate}
            \end{theorem}
            % 
            \begin{figure}[h!]
                \centering
                \subfigure[$\Delta < 0$]{
                   \begin{tikzpicture}
                        \begin{axis}[
                            xmin=-5,
                            xmax=5,
                            ymin=-7,
                            ymax=7,
                            xlabel={$x$},
                            ylabel={$y$},
                            scale only axis,
                            axis lines=middle,
                            domain=-2.1038:3,
                            samples=200,
                            smooth,
                            clip=false,
                            axis equal image=true,
                        ]
                            \addplot [red] {sqrt(x^3-3*x+3)}
                                node[right] {$y^2=x^3-3x+3$};
                            \addplot [red] {-sqrt(x^3-3*x+3)};
                        \end{axis}
                    \end{tikzpicture}
                }
                \subfigure[$\Delta > 0$]{
                    \begin{tikzpicture}
                        \begin{axis}[
                            xmin=-1.5,
                            xmax=1.5,
                            ymin=-2,
                            ymax=2,
                            xlabel={$x$},
                            ylabel={$y$},
                            scale only axis,
                            axis lines=middle,
                            domain=-1:0, 
                            samples=200,
                            smooth,
                            clip=false,
                            axis equal image=true,
                        ]
                            \addplot [red] {sqrt(x^3-x)};
                            \addplot [red] {-sqrt(x^3-x)};
                        \end{axis}
                        \begin{axis}[
                            xmin=-1.5,
                            xmax=1.5,
                            ymin=-2,
                            ymax=2,
                            xlabel={$x$},
                            ylabel={$y$},
                            scale only axis,
                            axis lines=middle,
                            domain=0.1:1.5, 
                            samples=200,
                            smooth,
                            clip=false,
                            axis equal image=true,
                        ]
                            \addplot [red] {sqrt(x^3-x)}
                                node[right] {$y^2=x^3-x$};
                            \addplot [red] {-sqrt(x^3-x)};
                        \end{axis}
                    \end{tikzpicture}
                } 
            \end{figure}
            \begin{figure}[h!]
                \centering
                \subfigure[Одна касательная. \\ $\Delta = 0 = c_4$]{
                    \begin{tikzpicture}
                        \begin{axis}[
                            xmin=-1.5,
                            xmax=5,
                            ymin=-5,
                            ymax=5,
                            xlabel={$x$},
                            ylabel={$y$},
                            scale only axis,
                            axis lines=middle,
                            domain=0:3, 
                            samples=200,
                            smooth,
                            clip=false,
                            axis equal image=true,
                        ]
                            \addplot [red] {sqrt(x^3)}
                                node[right] {$y^2=x^3$};
                            \addplot [red] {-sqrt(x^3)};
                        \end{axis}
                    \end{tikzpicture}
                }
                \subfigure[$\Delta < 0$]{
                    \begin{tikzpicture}
                        \begin{axis}[
                            xmin=-1.5,
                            xmax=1.5,
                            ymin=-1.5,
                            ymax=1.5,
                            xlabel={$x$},
                            ylabel={$y$},
                            scale only axis,
                            axis lines=middle,
                            domain=0:1, 
                            samples=200,
                            smooth,
                            clip=false,
                            axis equal image=true,
                        ]
                            \addplot [red] {sqrt(x^3+x)}
                                node[right] {$y^2=x^3+x$};
                            \addplot [red] {-sqrt(x^3+x)};
                        \end{axis}
                    \end{tikzpicture}
                }               
                \caption{Эллиптические кривые над $\R$} 
            \end{figure}
            
            \clearpage
            
        \subsection{Особые формы уравнения (\ref{eq:weierstrassequation})}
            \paragraph{$char K \neq 2$:}
            \begin{equation*}
                f: y^2+a_1xy + a_3y = x^3 + a_2x^2 + a_4x + a_6 \tag{\ref{eq:weierstrassequation}}
            \end{equation*}
            %
            \[
            \text{Дополним до полного квадрата: } y^2 + 2y(a_1x+a_3)+(a_1x+a_3))^2- \frac{1}{4}(a_1x+a_3)^2
            \]
            \[
            \implies 4(2y + \frac{a_1x+a_3}{2})^2 = 4 + \frac{1}{4}(a_1x+a_3)^2 + (a_2x^2 + a_4x + a_6) \quad| \cdot 4
            \]
            \[
            \implies (2y + a_1x+a_3)^2 = 4x^3 + (a_1^2 + 4a_3)x^2 + (2a_1a_3 + 4a_2)x^2 + a_3^2 + 4a_6
            \]
            \[
            \implies y = \frac{1}{2}(y' - a_1x - a_3)
            \]
            \[
            \implies y^2 = 4x^3 + d_2x^2 + 2d_4 + d_6
            \]
            \textit{Вывод: } $(x, y)\mapsto(x, \frac{1}{2}(y-a_1x-a_3))$ для $E/K, char K \neq 2$, преобразует кривую вида (\ref{eq:weierstrassequation}) к кривой
            \begin{equation}
            \label{eq:eOverKcharKNot2}
                E/K: y^2 = 4x^3 + d_2x^2 + 2d_4 + d_6.
            \end{equation}
            
            \paragraph{$char K \neq 2, 3$:} 
            Дополним правую часть (\ref{eq:eOverKcharKNot2}) до полного куба. Замена переменных
            \[
            (x, y) \mapsto \left(\frac{x-3d_2}{36}, \frac{y}{216}\right)
            \]
            Преобразует (\ref{eq:eOverKcharKNot2}) в 
            \begin{align}
                E/K: y^2 = x^3 + ax + b
            \end{align}
            \[
                a = -27c_4
            \]
            \[
                b = -56(d_2^3 + 36d_2d_4 - 216d_6) 
            \]
            В этом случае, 
            \begin{align*}
                \Delta &= -16(4a^3 + 27b^2) \\ \nonumber
                j(E) &= -1728 \frac{4a^3}{\Delta}. \nonumber
            \end{align*}
                  
            \paragraph{$char K = 2$:} 
            \[
            j(E)\neq0 \, (a_1\neq0) \implies (x, y) \mapsto (a_1^2x+\frac{a_3}{a_1}; \, a_1^3y + \frac{a_1^2a_4+a_3^2}{a_1^3})
            \]
            \begin{equation}
                E/K: y^2+xy=x^3+a_2'x^2+a_6'
            \end{equation}
            
            \[
            j(E)\neq0 \, (a_1\neq0) \implies (x, y) \mapsto (x+a_2, y)
            \]
            \begin{equation}
                E/K: y^2+a_3y = x^3+a_4x+a_6
            \end{equation}
            
        \subsection{Изоморфизм эллиптических кривых}
            \begin{definition} 
                $E_1/K, E_2/K$ изоморфны, если они изоморфны как проективные многообразия, т.е. $\exists$ морфизмы $\phi: E_1/K \to E_2/K, \psi: E_2/K \to E_1/K$ (определённые над $K$), такие что $\psi \circ \phi = id_{E_1}, \phi \circ \psi = id_{E_2}$.
            \end{definition}
            
            \begin{theorem}
                Пусть $E_1/K, E_2/K$ — две эллиптические кривые, заданные уравнениями 
                \begin{align}
                    E_1&: y^2+a_1xy + a_3y = x^3 + a_2x^2 + a_4x + a_6 \\ \nonumber
                    E_2&: y^2+a_1'xy + a_3'y = x^3 + a_2'x^2 + a_4'x + a_6'
                \end{align}
                \begin{equation*}
                E_1 \cong E_2 \iff \exists u,r,s,t \in K, u\neq0, \text{такие что замена}
                \end{equation*}
                \begin{equation}
                \label{eq:xytransition}
                    (x,y) \mapsto (u^2x+r, u^3y+ u^2sx+t)
                \end{equation}
                преобразует уравнение кривой $E_1$ в уравнение кривой $E_2$. Изоморфизм кривых задаёт отношение эквивалентности.
                \begin{align*}
                    \phi : (x,y)&\mapsto (u^{-2}(x-r), \,u^{-3}(y-sx-t+rs))\,\text{— точки $E_1$ в $E_2$} \\
                    \psi : (x,y)&\mapsto (u^2x+r, \,u^3y+u^2sx+t))\, \text{— точки $E_2$ в $E_1$} \\
                    \phi \circ \psi = i&d_{E_2}, \psi \circ \phi = id_{E_1}.
                \end{align*}
            \end{theorem}
            
            \begin{examples} $ $\newline
                Кривая $E$ из (4) $\cong$ (2), если $char K \neq 2$.\newline
                (5) $\cong$ (4) $\cong$ (2), если $char K \neq 2,3$.\newline
                (6) $\cong$ (2), если $char K = 2, j(E_1)\neq0$.\newline
                (7) $\cong$ (2), если $char K = 2, j(E_1)=0$
            \end{examples}
            
            \begin{corollary}
            \label{cor3}
                Если $E_1, E_2$ определены над $K$ с $char(K) \neq 2,3$, то (\ref{eq:xytransition}) можно упростить:
                \[
                    (x,y) \mapsto (u^2x, u^3y), u \neq 0
                \]
            \end{corollary}
            
            С помощью преобразования (9), можно вывести коэффициенты кривой $E_2$:
            \begin{align}
            \label{eq:e2transition}
                a_1' &= \frac{1}{u}(a_1+2s) \\  \nonumber
                a_2' &= \frac{1}{u^2}(a_2-sa_1+3r-s^2) \\ \nonumber
                a_3' &= \frac{1}{u^3}(a_3+ra_1+2t) \\ \nonumber
                a_4' &= \frac{1}{u^4}(a_4 = sa_3 + 2ra_2 = (t+rs)a_1 + 3r^2 - 2st) \\ \nonumber
                a_6' &= \frac{1}{u^6}(a_6 + ra_4 + r^2a_2 + r^3 - ta_3 - t^2 - rta_1) \\ \nonumber
            \end{align}
            
            Аналогично, можно получить уравнения для $d_i, j$:
            \begin{align*}
                \Delta' &= \frac{1}{u^{12}}\Delta \\
                j' &= j
            \end{align*}
            
            \begin{theorem}
                $E_1 \cong E_2$ над $\Bar{K} \iff j(E_1)=j(E_2)$
            \end{theorem}
            \begin{proof}
                Докажем для случая $char(K)\neq 2,3$ (см. \cite{silverman} для общего случая).
                
                $\implies$ следует из формул (\ref{eq:e2transition}).
                
                $\Longleftarrow$ Рассмотрим
                \begin{align*}
                    E_1&:y^2=x^3+ax+b \\
                    E_2&=(y')^2 = (x')^3+a'x'+b'
                \end{align*}
                Тогда из $j(E_1)=j(E_2) \Rightarrow$
                \begin{align*}
                    \frac{(4a)^3}{4a^3+27b^2} &= \frac{(4a')^3}{4(a')^3+27(b')^2} \\
                    4^4a^3(a')^3 + 4^3a^327(b')^2 &= 4^4 a'^3a^3 + 4^3(a')^327b^2 \\
                    a^3b^{12} &= a^{13}b^2 \tag{*}
                \end{align*}
                Нас интересуют только изоморфизмы вида $(x,y)\mapsto(u^2x', u^3y')$ (Следствие \ref{cor3}).

                Рассмотрим 3 случая: \newline
                \textbf{Случай 1.} $a = 0 \,(\Rightarrow j=0)$. Тогда $b\neq0$ (т.к. $\Delta \neq0$), $a'=0$;
                \begin{equation*}
                    y^2=x^3+b, \quad \quad y^{12} = x'^3 + b', \quad \quad u=\left(\frac{b}{b'}\right)^{\frac{1}{6}}
                \end{equation*}
                
                \textbf{Случай 2.} $b=0$ (Тогда $a \neq 0, b' = 0$).
                \[
                     u = \left(\frac{a}{a'}\right)^{\frac{1}{4}}
                \]
                
                \textbf{Случай 3.} $b\neq0 \implies a'b' \neq0$ 
                \[
                    u = \left(\frac{a}{a'}\right)^{\frac{1}{4}} = \left(\frac{b}{b'}\right)^{\frac{1}{6}}
                \]
            \end{proof}
            
            \textbf{Результаты}
            \begin{enumerate}
                \item Получить уравнение изоморфных для $E$ кривых над $K$, надо взять $(u,r,s,t)\in K$ и получить коэффициенты изоморфной кривой из (\ref{eq:e2transition}). \newline
                Сложность: $\bigO(1)$ операций деления/умножения в $K$.
                \item Проверить, являются ли две кривые $E_1/K, E_2/K$ изоморфными: решить (\ref{eq:e2transition}) для неизвестных $(u,r,s,t)$. Если решение в $K$ существует, значит $E_1\cong E_2$. \newline
                Сложность: полиномиальная -- решение системы уравнений в $K$.
                \item Определить, над каким полем $L \subseteq K$ изоморфны кривые $E_1, E_2$: в каком расширении $K$ лежат решения системы (\ref{eq:e2transition}).
            \end{enumerate}
            
	
\end{document}