\documentclass[11pt]{article}
\usepackage{amsmath,amssymb,amsthm}
\usepackage{algorithm}
\usepackage[noend]{algpseudocode} 

%---enable russian----

\usepackage[utf8]{inputenc}
\usepackage[russian]{babel}


%---tikz----
\usepackage{tikz}
\usetikzlibrary{arrows, chains, matrix, positioning, scopes, patterns, shapes}



% MACROS 
% PROBABILITY SYMBOLS
\newcommand*\PROB\Pr 
\DeclareMathOperator*{\EXPECT}{\mathbb{E}}


% GROUPS/DISTRIBUTIONS/SETS/LISTS
\newcommand{\N}{{{\mathbb N}}}
\newcommand{\Z}{{{\mathbb Z}}}
\newcommand{\Q}{{{\mathbb Q}}}
\newcommand{\R}{{{\mathbb R}}}
\newcommand{\F}{{{\mathbb F}}}
\newcommand{\PP}{{{\mathbb P}}}
\newcommand*{\IZ}{\ensuremath{\mathbb{Z}}}
\newcommand*{\IN}{\ensuremath{\mathbb{N}}}
\newcommand*{\IR}{{{\mathbb R}}}
\newcommand{\Zp}{\ints_p} % Integers modulo p
\newcommand{\Zq}{\ints_q} % Integers modulo q
\newcommand{\Zn}{\ints_N} % Integers modulo N
\newcommand{\Zr}{\ensuremath{\mathbb{Z}/r\mathbb{Z}}} % Integers modulo N
\newcommand*{\dDR}{\mathrm{d}} %de-Rham-Differential (the d in dx, dy, dz and so on)
\newcommand{\transpose}{\mkern0.1mu^{\mathsf{t}}}
\newcommand*{\union}{\mathbin{\cup}}

% Landau 
\newcommand{\bigO}{\mathcal{O}}
\newcommand*{\OLandau}{\bigO}
\newcommand*{\WLandau}{\Omega}
\newcommand*{\xOLandau}{\widetilde{\OLandau}}
\newcommand*{\xWLandau}{\widetilde{\WLandau}}
\newcommand*{\TLandau}{\Theta}
\newcommand*{\xTLandau}{\widetilde{\TLandau}}
\newcommand{\smallo}{o} %technically, an omicron
\newcommand{\softO}{\widetilde{\bigO}}
\newcommand{\wLandau}{\omega}
\newcommand{\negl}{\mathrm{negl}} 

% Misc
\newcommand{\eps}{\varepsilon}
\newcommand{\inprod}[1]{\left\langle #1 \right\rangle}

 
\newcommand{\handout}[5]{
  \noindent
  \begin{center}
  \framebox{
    \vbox{
      \hbox to 5.78in { {\bf  } \hfill #2 }
      \vspace{4mm}
      \hbox to 5.78in { {\Large \hfill #5  \hfill} }
      \vspace{2mm}
      \hbox to 5.78in { {\em #3 \hfill #4} }
    }
  }
  \end{center}
  \vspace*{4mm}
}

\newcommand{\lecture}[4]{\handout{#1}{#2}{#3}{Оформил #4}{Лекция #1}}

\newtheorem{theorem}{Теорема}
\newtheorem{corollary}[theorem]{Следствие}
\newtheorem{lemma}[theorem]{Лемма}
\newtheorem{observation}[theorem]{Observation}
\newtheorem{proposition}[theorem]{Предложение}

\theoremstyle{definition}
\newtheorem{definition}[theorem]{Определение}

\newtheorem{claim}[theorem]{Утверждение}
\newtheorem{fact}[theorem]{Факт}
\newtheorem{assumption}[theorem]{Предположение}

% 1-inch margins
\topmargin 0pt
\advance \topmargin by -\headheight
\advance \topmargin by -\headsep
\textheight 8.9in
\oddsidemargin 0pt
\evensidemargin \oddsidemargin
\marginparwidth 0.5in
\textwidth 6.5in

\parindent 0in
\parskip 1.5ex

\begin{document}
    
    \lecture{Номер лекции --- 06.09.2019}{Осень 2019}{Лектор: Елена Киршанова}{Филипп Максимов}
    
    
    \paragraph{Литература.}
        \begin{enumerate}
            \item A. Menezes "Elliptic curve public key cryptosystems"
            \item D. Hankerson, A. Menezes, S. Vanstone "Guide to elliptic curve cryptography"
            \item J. Silverman "Arithmetic of Elliptic Curves"
        \end{enumerate}
    
    \section{Введение}
    
        $\F_q$ — конечное поле, $|\F_q|=q=p^k$, $p$ — прсотое, $K$ — поле, $\Bar{K}$ — алгебраическое замыкание.
        
        \subsection{Определения}
        
            \begin{definition} Уравнение Вейерштрасса в проективных координатах 
            — уравнение степени 3 вида 
                \begin{equation}
                    F: Y^2Z + a_1 X Y Z + a_3 Y Z^2 = X^3 + a_2 X^2 Z + a_4 X Z^2 + a_6 Z^3
                \end{equation}
                где $a_i \in K$.
                Уравнение Вейерштрасса \textbf{гладкое} (или несингулярное), если для любых проективных точек $P=(X:Y:Z) \in \PP^2(K)$ \footnotemark[1], удовлетворяющих условию (1), хотя бы одна из частных производных $\frac{dF}{dX},\frac{dF}{dY},\frac{dF}{dZ}$ не обращается в $0$ на $P$. Если все три частных производные обращаются в $0$ хотя бы на одной точке $P$ (точке сингулярности), (1) — сингулярное уравнение.
                \footnotetext[1]{проективная плоскость над $K$ — множество классов эквивалентности на $K^3\setminus \{0,0,0\}$, т.е. $\overrightarrow{X} \sim \overrightarrow{Y}$, если $x_1=u*y_1,x_2=u*y_2,x_3=u*y_3$ }
            \end{definition}
            
            \begin{definition} 
                \textbf{Эллиптическая кривая} $E$ (алгебраическая кривая рода 1) — множество всех точек в $\PP^2(K)$, удовлетворяющих гладкой кривой (1). Существует всего одна точка в $E$ с координатой $Z=0: (0:1:0)$. Обозначаем эту точку $\bigO$, называем точкой в бесконечности. 
            \end{definition}
            
            \begin{definition} 
                Уравнение Вейерштрасса в аффинных координатах: 
                \[
                (x=X/Z, y=Y/Z)
                \]
                \begin{equation}
                    f: y^2+a_1xy + a_3y = x^3 + a_2x^2 + a_4x + a_6
                \end{equation}
                Тогда $F(K) = \{ (x,y) \in K \times K: f(x,y)=0 \} \union \{\bigO\}$.
                
                Если $a_i \in K \forall i$, то будем говорить, что кривая $E$ определена над $K$.
            \end{definition}
            
            \begin{definition}
                Обозначим
                \begin{align}
                    d_2 &= a_1^2 + 4a_2 \\ \nonumber
                    d_4 &= 2a_4 + a_1a_3 \\ \nonumber
                    d_6 &= a_3^2 + 4a_6 \\ \nonumber
                    d_8 &= a_1^2a_6 + 4a_2a_6 - a_1a_3a_4 + a_2a_3^2 - a_4^2 \\ \nonumber
                    c_4 &= d_2^2 - 24d_4 \\ \nonumber
                    \text{Для проверки: } 4d_8 &= d_2d_6 - d_4^2
                \end{align}
            \end{definition}
            
        \begin{theorem}
        	Утверждение теоремы
        \end{theorem}
        
        \begin{proof}
        	Доказательство
        \end{proof}
    
    \section{Алгоритм}
    
        \begin{algorithm}[ph]
        	\caption{Название алгоритма}
        	\label{alg:AlgName}
        	\textbf{Input:} Входные данные \\
        	\textbf{Output:} Выходные данные
        	
        	\begin{algorithmic}[1]
        		
        		\State 
        		\State{} \Return{} 
        	\end{algorithmic}
        
        \end{algorithm}
        
    
    \section{Анализ}
        \subsection{Blah blah blah}
        Текст лекции
        
        \subsubsection{Blah blah blah}
        Текст лекции $\bigO$
        

\end{document}