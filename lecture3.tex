\documentclass[12pt]{article}
\usepackage{amsmath,amssymb,amsthm}
\usepackage{algorithm}
\usepackage[noend]{algpseudocode} 
\usepackage{enumitem}

\usepackage{graphicx}

%---enable russian----

\usepackage[utf8]{inputenc}
\usepackage[russian]{babel}


%---tikz----
\usepackage{tikz}
\usetikzlibrary{arrows, chains, matrix, positioning, scopes, patterns, shapes}
\usepackage{pgfplots, subfigure}

\usepackage{biblatex}

% MACROS 
% PROBABILITY SYMBOLS
\newcommand*\PROB\Pr 
\DeclareMathOperator*{\EXPECT}{\mathbb{E}}


% GROUPS/DISTRIBUTIONS/SETS/LISTS
\newcommand{\N}{{{\mathbb N}}}
\newcommand{\Z}{{{\mathbb Z}}}
\newcommand{\Q}{{{\mathbb Q}}}
\newcommand{\R}{{{\mathbb R}}}
\newcommand{\F}{{{\mathbb F}}}
\newcommand{\PP}{{{\mathbb P}}}
\newcommand*{\IZ}{\ensuremath{\mathbb{Z}}}
\newcommand*{\IN}{\ensuremath{\mathbb{N}}}
\newcommand*{\IR}{{{\mathbb R}}}
\newcommand{\Zp}{\ints_p} % Integers modulo p
\newcommand{\Zq}{\ints_q} % Integers modulo q
\newcommand{\Zn}{\ints_N} % Integers modulo N
\newcommand{\Zr}{\ensuremath{\mathbb{Z}/r\mathbb{Z}}} % Integers modulo N
\newcommand*{\dDR}{\mathrm{d}} %de-Rham-Differential (the d in dx, dy, dz and so on)
\newcommand{\transpose}{\mkern0.1mu^{\mathsf{t}}}
\newcommand*{\union}{\mathbin{\cup}}

% Landau 
\newcommand{\bigO}{\mathcal{O}}
\newcommand*{\OLandau}{\bigO}
\newcommand*{\WLandau}{\Omega}
\newcommand*{\xOLandau}{\widetilde{\OLandau}}
\newcommand*{\xWLandau}{\widetilde{\WLandau}}
\newcommand*{\TLandau}{\Theta}
\newcommand*{\xTLandau}{\widetilde{\TLandau}}
\newcommand{\smallo}{o} %technically, an omicron
\newcommand{\softO}{\widetilde{\bigO}}
\newcommand{\wLandau}{\omega}
\newcommand{\negl}{\mathrm{negl}} 

% Misc
\newcommand{\eps}{\varepsilon}
\newcommand{\inprod}[1]{\left\langle #1 \right\rangle}

 
\newcommand{\handout}[5]{
  \noindent
  \begin{center}
  \framebox{
    \vbox{
      \hbox to 5.78in { {\bf  } \hfill #2 }
      \vspace{4mm}
      \hbox to 5.78in { {\Large \hfill #5  \hfill} }
      \vspace{2mm}
      \hbox to 5.78in { {\em #3 \hfill #4} }
    }
  }
  \end{center}
  \vspace*{4mm}
}

\newcommand{\lecture}[4]{\handout{#1}{#2}{#3}{Оформил #4}{Лекция #1}}

\newtheorem{theorem}{Теорема}
\newtheorem{corollary}[theorem]{Следствие}
\newtheorem{lemma}[theorem]{Лемма}
\newtheorem{observation}[theorem]{Observation}
\newtheorem{proposition}[theorem]{Предложение}

\theoremstyle{definition}
\newtheorem{definition}[theorem]{Определение}

\newtheorem{claim}[theorem]{Утверждение}
\newtheorem{fact}[theorem]{Факт}
\newtheorem{assumption}[theorem]{Предположение}

\theoremstyle{definition}
\newtheorem{examples}[theorem]{Примеры}

\theoremstyle{definition}
\newtheorem{example}[theorem]{Пример}

% 1-inch margins
\topmargin 0pt
\advance \topmargin by -\headheight
\advance \topmargin by -\headsep
\textheight 8.9in
\oddsidemargin 0pt
\evensidemargin \oddsidemargin
\marginparwidth 0.5in
\textwidth 6.5in

\parindent 0in
\parskip 1.5ex

%\addbibresource{books.bib}

\begin{document}
    
\lecture{№3 --- 27.09.19}{Осень 2019}{Лектор: Елена Киршанова}{Филипп Максимов}

\begin{definition}
	\label{def_01}
	Порядок точки $P \in E$, $\operatorname{ord} P, n \in \N$~---~минимальное, такое что
	$$
	n \cdot P = \mathcal{O}
	$$
\end{definition}

\section{Точки $n$-кручения}

\begin{definition}
	\label{def_02}
	Для $n > 1$, $E$~---~эллиптическая кривая над полем $K$ 
	$$
	E[n] = \left\{ {\left. P \in E(\bar K)\right|nP = \mathcal{O}} \right\}
	$$
	--- точки $n$-кручения.
\end{definition}

\underline{Рассмотрим $n=2$}

\begin{itemize}
	\item $charK \ne 2 \Rightarrow E \cdot {y^2} \pm f\left( x \right),\quad \deg f\left( x \right) = 3 \Rightarrow $
	$$
	{y^2} = \left( {x - {e_1}} \right)\left( {x - {e_2}} \right)\left( {x - {e_3}} \right),\quad {e_i} - {\text{ корни }}f\left( x \right){\text{ в }}\bar{K}
	$$
	
	Для $\forall P \in E$ справедливо: $2P = \mathcal{O} \Leftrightarrow $ касательная $l$ в $P$~---~вертикальная $ \Rightarrow y = 0$ $ \Rightarrow $
	$$
	E\left[ 2 \right] = \left\{ \mathcal{O},\;\left( {{e_1},0} \right),\;\left( {{e_2},0} \right),\;\left( {{e_3},0} \right) \right\} \cong {\Z_2} \oplus {\Z_2}.
	$$
	
	Вывод: для того, чтобы найти все точки $r$-кручения, в $charK \ne 2$, следует найти все корни $f(x)$.
	
	\item $charK = 2: \exists\: 2\text{ вида кривой }A$

    \begin{center}
	\begin{tabular}{c c}
		$ \downarrow $ & $ \downarrow $ \\
		$E: {y^2} + xy + {x^3} + {a_2}{x^2} + {a_6} = 0$ \quad & $E:{y^2} + {a_3}y + {x^3} + {a_4}x + {a_6}$ \\
		$\left( {{a_6} \ne 0} \right)$ & $\left( {{a_3} \ne 0} \right)$ \\
	\end{tabular}
    \end{center}

	В обоих случаях, если $P = \left( {x,y} \right)$~---~точка порядка 2, то касательная к $P$~---~вертикаль $ \Rightarrow \frac{{dE}}{{dy}} = 0$

	\begin{center}
	\begin{tabular}{c c}
		$ \downarrow $ & $ \downarrow $ \\
		$2y \ne x = 0$ & $\frac{{dE}}{{dy}} = {a_3}$ \\
		$x = 0$ & ${a_3} \ne 0$ (иначе $E$ — сингулярная) \\
		$ \Rightarrow {y^2} + {a_6} = 0$ & $ \Rightarrow E\left[ 2 \right] = \left\{\mathcal{O}\right\}.$ \\
		$ \Rightarrow P = \left( {0, \sqrt {{a_6}} } \right)$ & \\
		$ \Rightarrow E\left[ 2 \right] = \left\{ {\mathcal{O},\;\left( {0,\sqrt {a_6} } \right)} \right\} \simeq {\Z_2}$ & \\
	\end{tabular}
	\end{center}
\end{itemize}

\begin{lemma}
	\label{lemm_01}
	Для $E$~---~эллиптической кривая над $K$, справедливо
	\begin{align*}
	E\left[ 2 \right] &\cong {\Z_2} \oplus {\Z_2}, &\text{ при }charK \ne 2 \\
	E\left[ 2 \right] &\cong 0 \text{ либо } E\left[ 2 \right] \cong {\Z_2} & \text{ при } char K = 2.
	\end{align*}

\end{lemma}

Можно показать, что [Was. \S~3.1]
    \begin{align*}
    E\left[ 3 \right] &\cong {\Z_3} \oplus {\Z_3}, & {\text{ при }}char K \ne 3 \\
    E\left[ 3 \right] &\cong 0 \text{ либо } E\left[ 3 \right] \cong \Z_3, & {\text{ при }} char K = 3
    \end{align*}

В общем случае, справедлива теорема \ref{theor_02}:

\begin{theorem}
	\label{theor_02}
	Пусть $E$~---~эллиптическая кривая над $K$, и $n \geqslant {\N_+}$. Тогда справедливо: [док-во в Was. \S~32]
	\begin{itemize}
		\item $E\left[ n \right] \cong {\Z_n} \oplus {\Z_n},{\text{ если }} char K \nmid n{\text{ или }}char K \ne 0,$
		
		\item $E\left[ n \right] \cong {\Z_{n'}} \oplus {\Z_{n'}}{\text{ или }} \cong {\Z_n} \oplus {\Z_{n'}}{\text{ если }}char\;K = p > 0,\quad p|n$
		и $n = p^r \cdot n',\quad p \nmid n'.$
	\end{itemize}
\end{theorem}

\begin{definition}
	\label{def_03}
	$ $
	\begin{itemize}
	\item $E$, заданная над $K$ с $char\;K = p$, называется \underline{простой}, если $E[p] \cong {\Z_p}$.

	\item $E$ называется \underline{суперсингулярной}, если $E[p] \cong 0$.\\
	! Не путать с сингулярными кривыми.
	\end{itemize}
\end{definition}

\section{Многочлены деления}

Важность:
\begin{itemize}
	\item [--] описывают отображение $ n: P \leftrightarrow n \cdot P$
	\item [--] используются в алгоритме подсчета точек кривой
	\item [--] используются в вычислениях изогений
\end{itemize}

\begin{definition}
	\label{def_04}
	$A,B,x,y$~---~переменные.
	
	Многочлены деления ${\psi _m} \in \Z\;\left[ {x,y,A,B} \right]$ определяются рекуррентными соотношениями:
    \begin{align*}
	{\psi _0} &= 0 \\
	{\psi _1} &= 1 \\
	{\psi _2} &= 2y \\
	{\psi _3} &= 3{x^4} + 6A{x^2} + 12Bx - {A^2} \\
	{\psi _4} &= 4{y^2}\left( {{x^6} + 5A{x^4} + 20B{x^3} - 5{A^2}{x^2} - 4ABx - z{B^2} - {A^3}} \right) \\
	{\psi _{2m + 1}} &= {\psi _{m + 2}}\psi _m^3 - {\psi _{m - 1}}\psi _{m + 1}^3, \quad m \geqslant 2 \\
	{\psi _{2m}} &= {\left( {2y} \right)^{ - 1}} \cdot {\psi _m} \cdot \left( {{\psi _{m + 2}}\psi _{m - 1}^2 - {\psi _{m - 2}}\psi _{m + 1}^2} \right),\quad m \geqslant 3
	\end{align*}
\end{definition}

Могут быть получены с помощью формул сложения в так называемых координатах Якоби.

\underline{Свойства} (док-во: Was. Lemma 3.3)

\begin{enumerate}
	\item 
    	${\psi _n} \in \Z[x, y^2, A, B]$ если $n$ — нечетное \\
    	${\psi _n} \in 2y\Z[x, y^2, A, B]$, если $n$ — четное. 
	\item Определим
    	\begin{align*}
    	{\varphi _m} &= x \cdot \psi _m^2 - {\psi _{m + 1}}{\psi _{m - 1}} \\
    	{\omega _m} &= {\left( {4y} \right)^{ - 1}}\left( {{\psi _{m + 2}}\psi _{m - 1}^2 - {\psi _{m - 2}}\psi _{m + 1}^2} \right) \\
    	{\varphi _n} &\in \Z\left[ {x, {y^2},A,B} \right], \forall n \\
    	{\omega _n} &\in y\Z\;\left[ {x, {y^2},A,B} \right], n{\text{  --  нечетное}} \\
    	{\omega _n} &\in \Z\left[ {x, {y^2},A,B} \right],n{\text{ -- четное}}
    	\end{align*}

	
	\item Для эллиптической кривой $E: {y^2} = {x^3} + Ax + B$, в многочленах ${\psi _n}$, ${\phi _n}$ можно сделать замену ${y^2} \mapsto {x^3} + Ax + B$ и рассматривать их как многочлены от $x$ (в $\Z\left[ {x,A,B} \right]$). Тогда 
    	\begin{align*}
    	{\varphi _n}\left( x \right) &= {X^{{n^2}}} + {\text{ мономы степени }} < {n^2} \\
    	\psi _n^2\left( x \right) &= {n^2}{x^{{n^2} - 1}} + {\text{ мономы степени }} < {n^2} - 1
    	\end{align*}

\end{enumerate}

\begin{theorem}
	\label{theor_03}
	Пусть $E: {y^2} = {x^3} + Ax + B$ ($ \Rightarrow char X \neq 2,3$) $P = \left( {x,y} \right) \in E$, $n \in {\N_+ }$. Тогда
	$$nP = \left( {\frac{{{\phi _n}\left( x \right)}}{{\psi _n^2\left( x \right)}},\;\frac{{{\omega _n}\left( x \right)}}{{{{\left( {{\psi _n}\left( {x,y} \right)} \right)}^3}}}} \right)
	$$
\end{theorem}

Таким образом, отображение (эндоморфизм) <<умножение на $n$>> $n \cdot P$ задается рациональными функциями.
\end{document}

