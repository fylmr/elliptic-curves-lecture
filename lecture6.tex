\documentclass[12pt]{article}
\usepackage{amsmath,amssymb,amsthm}
\usepackage{mathtools}
\usepackage{algorithm}
\usepackage[noend]{algpseudocode} 
\usepackage{enumitem}

\usepackage{graphicx}

%---enable russian----

\usepackage[utf8]{inputenc} 
\usepackage[russian]{babel}


%---tikz----
\usepackage{tikz}
\usetikzlibrary{arrows, chains, matrix, positioning, scopes, patterns, shapes}
\usepackage{pgfplots, subfigure}

\usepackage{biblatex}

% MACROS 
% PROBABILITY SYMBOLS
\newcommand*\PROB\Pr 
\DeclareMathOperator*{\EXPECT}{\mathbb{E}}


% GROUPS/DISTRIBUTIONS/SETS/LISTS
\newcommand{\N}{{{\mathbb N}}}
\newcommand{\Z}{{{\mathbb Z}}}
\newcommand{\Q}{{{\mathbb Q}}}
\newcommand{\R}{{{\mathbb R}}}
\newcommand{\F}{{{\mathbb F}}}
\newcommand{\Fqd}{{{\F_{q^{d_i}}}}}
\newcommand{\PP}{{{\mathbb P}}}
\newcommand*{\IZ}{\ensuremath{\mathbb{Z}}}
\newcommand*{\IN}{\ensuremath{\mathbb{N}}}
\newcommand*{\IR}{{{\mathbb R}}}
\newcommand{\Zp}{\ints_p} % Integers modulo p
\newcommand{\Zq}{\ints_q} % Integers modulo q
\newcommand{\Zn}{\ints_N} % Integers modulo N
\newcommand{\Zr}{\ensuremath{\mathbb{Z}/r\mathbb{Z}}} % Integers modulo N
\newcommand*{\dDR}{\mathrm{d}} %de-Rham-Differential (the d in dx, dy, dz and so on)
\newcommand{\transpose}{\mkern0.1mu^{\mathsf{t}}}
\newcommand*{\union}{\mathbin{\cup}}

% Landau 
\newcommand{\bigO}{\mathcal{O}}
\newcommand*{\OLandau}{\bigO}
\newcommand*{\WLandau}{\Omega}
\newcommand*{\xOLandau}{\widetilde{\OLandau}}
\newcommand*{\xWLandau}{\widetilde{\WLandau}}
\newcommand*{\TLandau}{\Theta}
\newcommand*{\xTLandau}{\widetilde{\TLandau}}
\newcommand{\smallo}{o} %technically, an omicron
\newcommand{\softO}{\widetilde{\bigO}}
\newcommand{\wLandau}{\omega}
\newcommand{\negl}{\mathrm{negl}} 

% Misc
\newcommand{\eps}{\varepsilon}
\newcommand{\inprod}[1]{\left\langle #1 \right\rangle}
\DeclarePairedDelimiter{\abs}{\lvert}{\rvert}

%proper overline reduced by some mu's, re-adjust if needed
\newcommand{\overbar}[1]{\overline{\mkern-0.0mu#1\mkern-4.0mu}}

 
\newcommand{\handout}[5]{
  \noindent
  \begin{center}
  \framebox{
    \vbox{
      \hbox to 5.78in { {\bf  } \hfill #2 }
      \vspace{4mm}
      \hbox to 5.78in { {\Large \hfill #5  \hfill} }
      \vspace{2mm}
      \hbox to 5.78in { {\em #3 \hfill #4} }
    }
  }
  \end{center}
  \vspace*{4mm}
}

\newcommand{\lecture}[4]{\handout{#1}{#2}{#3}{Оформил #4}{Лекция #1}}

\newtheorem{theorem}{Теорема}
\newtheorem{corollary}[theorem]{Следствие}
\newtheorem{lemma}[theorem]{Лемма}
\newtheorem{observation}[theorem]{Observation}
\newtheorem{proposition}[theorem]{Предложение}

\theoremstyle{definition}
\newtheorem{definition}[theorem]{Определение}

\newtheorem{claim}[theorem]{Утверждение}
\newtheorem{fact}[theorem]{Факт}
\newtheorem{assumption}[theorem]{Предположение}

\theoremstyle{definition}
\newtheorem{examples}[theorem]{Примеры}

\theoremstyle{definition}
\newtheorem{example}[theorem]{Пример}
\newtheorem{remark}[theorem]{Замечание}

% 1-inch margins
\topmargin 0pt
\advance \topmargin by -\headheight
\advance \topmargin by -\headsep
\textheight 8.9in
\oddsidemargin 0pt
\evensidemargin \oddsidemargin
\marginparwidth 0.5in
\textwidth 6.5in

\parindent 0in
\parskip 1.5ex

%\addbibresource{books.bib}

\begin{document}
    % $(\frac{\cdot}{\F_q^{d_i}})$
\lecture{№6 — 25.10.19}{Осень 2019}{Лектор: Елена Киршанова}{Филипп Максимов}

\section{Алгоритм подсчёт $\F_q$-рациональных точек эллиптической кривой}

Из предыдущей лекции:\\
$ E(\F_q): y^2 = x^3 + Ax + B $ — эллиптическая кривая.

$\#E(\F_q) = \abs{E(\F_q)}$ — число $q$-рациональных точек кривой (или порядок кривой).

Теорема Хассе даёт верхнюю и нижнюю границы оценки для $\abs{E(\F_q)}$:
\[
    \abs{q+1-\#E(\F_q)} \leq 2 \sqrt{q}
\]

Асимптотика: $\#E(\F_q) = \bigO(q)$.
%Если $a = q + 1 - \#E(\F_q)$, то $\abs{a} \leq 2\sqrt{q}$, $a$ — след Фробениуса.

Мы доказали, что $\#E(\F_q) = q + 1 + \sum_{x \in \F_q}\left(\dfrac{x^3 + Ax+ B}{\F_q}\right)$, где $\left(\dfrac{\cdot}{\F_q}\right)$ — кв. вычет в $\F_q$ (если $q$ — простое $\Rightarrow$ символ Лежандра) \\
Время вычисления квадратичного вычета в таком случае: $\bigO(q \log q) = \softO(q)$ — экспоненциальное от числа бит $q$.

\subsection{Алгоритм подсчёта точек Baby Step — Giant Step \\ = Алгоритм нахождения порядка точки $P\in E(\F_q)$} 

\underline{Идея}: Пусть $N = \#E(\F_q)$ — неизвестно.

По теореме Лагранжа: $N\cdot P = \infty \;\; \forall P$

Так как $q+1-2\sqrt{q} \leq N \leq q+1+2\sqrt{q}$ (т.е.\ $N$ лежит в интервале р-ра $4\sqrt{q}$,
мы можем проверить все $N$ из этого интервала: $N\cdot P \stackrel{?}{=}
 \infty$).

Наивный метод (брутфорс): $\bigO(\sqrt{q})$\\
Алгоритм BS-GS -- стандартный алгоритм нахождения коллизий / цикла функции-- ускоряет до $\bigO(q^{\frac{1}{4}})$

Для начала опишем алгоритм нахождения порядка точки $P\in E$.

\subsubsection{Алгоритм BS—GS для нахождения порядока точки $P$}

\underline{Вход}: $P\in E(\F_q)$\\
\underline{Выход}: $k = \operatorname{ord}(P)$

1. $Q = (q+1)P$

2. (BS) Выберем минимальное $m \in \Z$: $m > q^\frac{1}{4}$\\
Вычислим и сохраним (в список $L$) все $\pm j \cdot P, j \in \{ 0, \ldots,  m\}$. Сортируем $L$.

3. (GS) Вычисляем точки $Q + k(2mP)$ для всех $k \in  \{-m \ldots m\}$, пока не найдём в списке $L$ точку $\pm  jP$, что 
\[
    Q + k(2mP) = \pm  jP.
\]
$Q + k(2mP) = \pm jP \iff (q+1 + 2mk \mp  j)P = \infty$.

4. Положим $M = q+1+2mK \mp  j$ (порядок $P$ — делитель $M$)

5. Факторизуем $M = p_1^{e_1}..p_r^{e_r}$

6. Для $i = 1..r$:

Вычислим $\left(\frac{M}{p_i} \right)$

Если $\left(\frac{M}{p_i} \right)P=\bigO$:

\quad $M \leftarrow \frac{M}{p_i}$

\quad Вернуться к Шагу 5 (неоптимально, но корректно). 

Если $\frac{M}{p_i}p_i \neq \infty \; \forall i$:

\quad Вернуть $M$.
 
Иначе

\quad Вернуть `fail'.

\subsubsection{Корректность}

1. Найдём ли мы коллизию на шаге 3?
\begin{lemma}[Разбиение элемента $x$ на $2m$ старших бит и $(\log x - 2m)$ младших]
    Пусть $x \in \Z: \abs{x} \leq 2m^2$. Тогда $\exists \ x_0, x_1 \in \Z$, т.ч. $-m < x_0 \leq m, -m < x_1 \leq m$
\end{lemma}
\begin{proof}
    Пусть
    \begin{align*}
        x_0 &= x \bmod 2m \text{, где} \bmod B \in \left(-\dfrac{B}{2}, \dfrac{B}{2} \right] \\
        x_1 &= \frac{x-x_0}{2m}
    \end{align*}
    
    Тогда $\abs{x_1} \leq \dfrac{2m^2 + m}{2m} < m+1$.
\end{proof}

2. Почему шаг 6 возвращает порядок $P$?
\begin{lemma}
    Пусть $G$ — аддитивная группа, $g \in G$. Положим $M > 0$, что $Mg = 0$, $M = p_1^{e_1}...p_r^{e_r}$, где $p_i$ — различные простые. Тогда если $\left(\dfrac{M}{p_i}\right)g \neq 0\ \forall i \in \{1..r\}$, то $M$ — порядок $g$.
\end{lemma}
\begin{proof}
    Пусть $k$ — порядок $g$. Тогда $k\ |\ M$. Положим $k \neq M$ (от противного).
    
    Пусть $p_i$ — простое делящее $\frac{M}{k}$.
    
    Тогда $(p_i \cdot k)\ |\ M$, или $k\ |\ \left(\frac{M}{p_i}\right) \Rightarrow \left(\frac{M}{p_i}\right)g = 0$ (т.е.\ мы нашли $p_i: \left(\frac{M}{p_i}\right)g = 0$), что противоречит утверждению теоремы $\Rightarrow k = M$.\\
\end{proof}

\subsubsection{Анализ сложности алгоритма}

\underline{Шаг 1} (Быстрое сложение)

$\bigO(\log q)$ операций <<+>> на кривой, каждый <<+>>: $\operatorname{poly lg} q \Rightarrow \bigO(\operatorname{poly lg} q)$

\underline{Шаг 2}

$\softO(m) = \softO(q^{1/4})$ — времени \\
${\bigO}(q^{1/4})$ — память.

\underline{Шаг 3}

$\softO(2m) = \softO(q^{1/4})$ — ожидаемое количество переборов $k$.

\underline{Шаг 4}

Элементарные операции в $\F_q$

\underline{Шаг 5}

$L\left[\frac{1}{3}, \frac{64}{9}^{\frac{1}{3}}\right] = \exp{\left(\left[\frac{64}{9}^{\frac{1}{3}} + \bigO(1)\right] (\log n)^{\frac{1}{3}} (\log \log q)^{\frac{2}{3}} \right)}$

\underline{Шаг 6}

$\softO(\sqrt{\log M} \cdot \operatorname{poly} \log q) = \softO(\sqrt{\log M}) = \softO(\operatorname{poly} \log q)$\\
макс. $r \approx \sqrt{\log M}$

\textbf{Итого}: самый затратный шаг — №3,\ $\softO(q^{\frac{1}{4}})$

\subsubsection{Замечания к алгоритму}
\begin{enumerate}
    \item Для оптимизации памяти (вычислений) на шаге 3 достаточно хранить\\ $x$-координату.
    \item Классическим алгоритмом cycle finding (Поллард-$\rho$) можно реализовать алгоритм, используя только $\operatorname{poly} \log q$ памяти
\end{enumerate}

\subsection{Алгоритм вычисления $\#(\F_q)$}
1. Выбрать $P \in E(\F_q)$ (случайно выбираем $x$-координату, посчитать $y: y^2 = x^3 + Ax+B$)

2. Найти $k_p = \operatorname{ord}P$

3. Повторить шаги (1), (2), получив множество порядков $\{k_{p_i} = \operatorname{ord} p_i\}_{i\leq t}$

for i = 1..N:

$\quad L = lcm(L, ord(P_i))$

\quad if $L \geq q+1-2\sqrt{q}$:\quad\quad \textit{(т.е. $L \in \left[q+1-2\sqrt{q}, q+1+2\sqrt{q} \right]$)}

\quad \quad return $L$

\quad if $L \geq 4\sqrt{q}$:

\quad \quad $M = q+1+[2\sqrt{q}] // L$

\quad \quad return $M\cdot L$

return $'fail'$

Всевозможные порядки, которые могут быть получены на шаге 3:
\[
    \operatorname{ord} p_i \in \{\prod_{0 \leq \ell_i \leq e_i} p_i^{e_i}\} \text{(все делители } \#E(\F_q)) \text{, где } \#E(\F_q) = \prod p_i^{e_i}
\]
Всего $\prod_{i=1}^r e_i$ всевозможных порядков, примерно $\softO(\log \#E(\F_q))$ повторов на шаге 3.

\section{Алгоритм Схофа (Schoof's Algorithm)}
\textit{R. Schoof "Counting points on elliptic curves over finite fields"}

$E: y^2 = x^3 + Ax + B$

Первый полиномиальный алгоритм от $\log q$(!), $q$ — простое. 
Основная идея: вычислить $\#E(\F_q) \operatorname{mod} \{2, 3 ... p\}$ — простые числа и затем восстановить $\#E(\F_q)$ по CRT.

\textbf{1.} Вычисление $\#E(\F_q) \operatorname{mod} 2: \#E(\F_q)$ — чётно \\
$\Leftrightarrow E(\F_q)$ содержит точку $\neq \infty$ порядка 2. \\
Точки порядка 2 имеют $y$-координату $= 0 \Leftrightarrow x^3 + Ax+B = 0$ в $\F_q$

Как определить, есть ли у $x^3 + Ax+ B$ корни в $\F_q$?\\
Все элементы $\F_q$ — корни $x^q-x$. Т.е. $x^3+Ax+B=0$ в $\F_q \Leftrightarrow \gcd(x^q - x, x^3+Ax+B) \neq 1$ в $\F_p[x]$

$\exists$ эффективный алгоритм (вычисление $x^q$ проводится в $\F_p[x] / (x^3 + Ax+ B)$ ускоренным алгоритмом возведения в степень.

Сложность: $\bigO(\log^3 q)$


\textbf{2.} Обобщим на другие $\ell \in \{2..p\}, \ell \neq 2$

Так как. $|\#E(\F_q) - q - 1| < 2\sqrt{q}$, для получения $\#E(\F_q)$ достаточно рассмотреть простые $\ell$, что 
\[
    \prod_\ell \ell > 4\sqrt{q}
\]
По теореме о распределении простых чисел $(p_n \sim n \log n)$, нам будет достаточно взять $\bigO(\log q)$ простых $\ell$, каждый размером $\bigO(\log q)$

\textbf{2.1.} Как и в случае $\ell = 2$, рассмотрим группу точек $\ell$-кручения 
\[
    E[\ell] = \{ P \in E(\overbar{\F_q}): \ell\cdot P = \bigO \} \simeq \Z/\ell\Z \times \Z/\ell\Z,
\]
где $\psi_q(x) \in \F_q[x] - \ell$ — многочлены деления ($\in \F_q[x]$ т.к. мы берём простые (нечётные) $\ell$). Они могут быть эффективно получены в явном виде с помощью рекуррентных соотношений.

\textbf{2.2.} Эндоморфизм Фробениуса $\phi_q: E \rightarrow E, (x,y)\mapsto (x^q, y^q)$ удовлетворяет соотношению (см. Лекцию №5)
\begin{equation}
\label{eq_1}
    \phi_q^2 - a\phi + q = 0, \text{ где } a \text{ — след Фробениуса } (a = q+1-\#E(\F_q))
\end{equation}
\[
    \text{(т.е. } (x^{q^2}, y^{q^2}) - [a](x^q, y^q) + [q] = \infty)
\]

Соотношение (\ref{eq_1}) справедливо и $\bmod \ell$, т.е.\\
$\phi_q^2 - a'\phi + q' = 0$ для некоторых $a' \in \{0 \bmod  \ell-1\}$,\\
что $a \equiv a' \bmod \ell, q' \in \{0 \ldots \ell-1\}, q=q' \bmod \ell$

Важно: соотношение (\ref{eq_1}) может быть задано с помощью многочленов $\Rightarrow$ для кандидатов $a'\ \exists$ эффективная проверка:
\[
    q'\cdot P = \left(\frac{\phi_{q'}(x)}{\psi^2_{q'}(x)}, \frac{\omega_{q'}(x,y)}{\psi^3_n(x,y)} \right)
\]
\[
    (x^{q^2}, y^{q^2}) + q'(X,Y) \equiv a'(x^q, y^q) \operatorname{mod} \psi_\ell(x) \text{ и } \bmod  (Y^2-X^3-AX-B)
\]

\subsection{Сложность алгоритма}

Для фиксированного $\ell\in\{2 \ldots p\}$ (повт. для каждого $\ell$, всего различных $\bigO(\log q)$):
\begin{itemize}
    \item Подсчёт $x^{q^2}, x^q, ... \bmod \psi_\ell(x)$\\
    $\bigO(\log q \cdot (\ell^2 \log q)^2) = $ возведение в степень $\times$ количество бит в многочлене
    
    \item Умножение точки $(x^q, y^q) a' \leq \ell$ раз для каждого кандидата $a:$\\
    $\bigO(\ell\cdot(\ell^2 \log q)^2)$
\end{itemize}
Всего: $\bigO(\log q \cdot(\bigO(\log q (\ell^2 \log q)^2) + \bigO(\ell(\ell^2 \log q)^2))) = \{\ell=\log q\} = \bigO(\log^8 q)$

\end{document}