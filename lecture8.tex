\documentclass[12pt]{article}
\usepackage{amsmath,amssymb,amsthm}
\usepackage{mathtools}
\usepackage{algorithm}
\usepackage[noend]{algpseudocode} 
\usepackage{enumitem}

\usepackage{graphicx}

%---enable russian----

\usepackage[utf8]{inputenc} 
\usepackage[russian]{babel}


%---tikz----
\usepackage{tikz}
\usetikzlibrary{arrows, chains, matrix, positioning, scopes, patterns, shapes}
\usepackage{pgfplots, subfigure}

\usepackage{biblatex}

% MACROS 
% PROBABILITY SYMBOLS
\newcommand*\PROB\Pr 
\DeclareMathOperator*{\EXPECT}{\mathbb{E}}


% GROUPS/DISTRIBUTIONS/SETS/LISTS
\newcommand{\N}{{{\mathbb N}}}
\newcommand{\Z}{{{\mathbb Z}}}
\newcommand{\Q}{{{\mathbb Q}}}
\newcommand{\R}{{{\mathbb R}}}
\newcommand{\F}{{{\mathbb F}}}
\newcommand{\Fqd}{{{\F_{q^{d_i}}}}}
\newcommand{\PP}{{{\mathbb P}}}
\newcommand*{\IZ}{\ensuremath{\mathbb{Z}}}
\newcommand*{\IN}{\ensuremath{\N}}
\newcommand*{\IR}{{{\mathbb R}}}
\newcommand{\Zp}{\ints_p} % Integers modulo p
\newcommand{\Zq}{\ints_q} % Integers modulo q
\newcommand{\Zn}{\ints_N} % Integers modulo N
\newcommand{\Zr}{\ensuremath{\mathbb{Z}/r\mathbb{Z}}} % Integers modulo N
\newcommand*{\dDR}{\mathrm{d}} %de-Rham-Differential (the d in dx, dy, dz and so on)
\newcommand{\transpose}{\mkern0.1mu^{\mathsf{t}}}
\newcommand*{\union}{\mathbin{\cup}}

% Landau 
\newcommand{\bigO}{\mathcal{O}}
\newcommand*{\OLandau}{\bigO}
\newcommand*{\WLandau}{\Omega}
\newcommand*{\xOLandau}{\widetilde{\OLandau}}
\newcommand*{\xWLandau}{\widetilde{\WLandau}}
\newcommand*{\TLandau}{\Theta}
\newcommand*{\xTLandau}{\widetilde{\TLandau}}
\newcommand{\smallo}{o} %technically, an omicron
\newcommand{\softO}{\widetilde{\bigO}}
\newcommand{\wLandau}{\omega}
\newcommand{\negl}{\mathrm{negl}} 

% Misc
\newcommand{\eps}{\varepsilon}
\newcommand{\inprod}[1]{\left\langle #1 \right\rangle}
\DeclarePairedDelimiter{\abs}{\lvert}{\rvert}

%proper overline reduced by some mu's, re-adjust if needed
\newcommand{\overbar}[1]{\overline{\mkern-0.0mu#1\mkern-4.0mu}}

 
\newcommand{\handout}[5]{
  \noindent
  \begin{center}
  \framebox{
    \vbox{
      \hbox to 5.78in { {\bf  } \hfill #2 }
      \vspace{4mm}
      \hbox to 5.78in { {\Large \hfill #5  \hfill} }
      \vspace{2mm}
      \hbox to 5.78in { {\em #3 \hfill #4} }
    }
  }
  \end{center}
  \vspace*{4mm}
}

\newcommand{\lecture}[4]{\handout{#1}{#2}{#3}{Оформил #4}{Лекция #1}}

\newtheorem{theorem}{Теорема}
\newtheorem{corollary}[theorem]{Следствие}
\newtheorem{lemma}[theorem]{Лемма}
\newtheorem{observation}[theorem]{Observation}
\newtheorem{proposition}[theorem]{Предложение}

\theoremstyle{definition}
\newtheorem{definition}[theorem]{Определение}

\newtheorem{claim}[theorem]{Утверждение}
\newtheorem{fact}[theorem]{Факт}
\newtheorem{assumption}[theorem]{Предположение}

\theoremstyle{definition}
\newtheorem{examples}[theorem]{Примеры}

\theoremstyle{definition}
\newtheorem{example}[theorem]{Пример}
\newtheorem{remark}[theorem]{Замечание}

% 1-inch margins
\topmargin 0pt
\advance \topmargin by -\headheight
\advance \topmargin by -\headsep
\textheight 8.9in
\oddsidemargin 0pt
\evensidemargin \oddsidemargin
\marginparwidth 0.5in
\textwidth 6.5in

\parindent 0in
\parskip 1.5ex


\begin{document}
\lecture{№8}{Осень 2019}{Лектор: Елена Киршанова}{Филипп Максимов}

\section{Тест на простоту Миллера—Рабина}

$a^{p-1} \equiv 1\bmod p,\; p$ — простое (Теорема Ферма)

$p-1 = 2^k\cdot q,\; q$ — нечётное. Тогда для $a$, т.ч. $p \nmid a$, либо:
1)\ $a^q \equiv 1\bmod p$ \\
2)\ $\text{одно из чисел } a^q, a^{2q}, ..., a^{2^{k-1}q} \equiv -1\bmod p$

$a^q, a^{2q} ... a^{2^k q} \equiv 1\bmod p$, все предыдущие числа в списке — квадраты друг друга.

Тогда либо первое число в списке $a^q \equiv 1\bmod p$ (и все остальные $\equiv 1\bmod p$), либо найдётся $b$ в списке, т.ч. \\
$b \not\equiv 1$ и $b^2 \equiv 1\bmod p$, т.е. $b \equiv -1\bmod p$

Если $\exists\ a$, т.ч. $\gcd(a,n)=1$ и оба условия 
\begin{equation*}
  \begin{cases} 
    a^q \not\equiv 1\bmod n\\
    a^{a^i q} \equiv 1\bmod n\; \forall i = 0...k-1
  \end{cases}
\end{equation*}
то $a$ — \textit{свидетель}, что $n$ — составное.\\

\begin{algorithm}[ph] 
	\caption{Miller—Rabin}
	\label{alg:MillerRabin}
	\textbf{Input:} n, a 
	
	\begin{algorithmic}[1]
		
		\State $n-1 = 2^kq,\ q$ — нечётное
		\State $a := a^q\bmod n$
		\If {$a \equiv 1\bmod n$} 
		    \State \Return $\bot$  
	    \EndIf
	    \For {i = 0..k-1}
	        \If{$a\equiv 1\bmod n$}
	            \State \Return $\bot$
	        \EndIf
	        \State $a := a^2\bmod n$
	    \EndFor
	    \State \Return «$n$ — составное»
	\end{algorithmic}
\end{algorithm}

Алгоритм повторяется $k$ раз для случайно выбранных $a \in [2..n-2]$

Время работы $\bigO(k \log^3 n)$, если используется быстрое возведение в степень.

Веронтность ошибки (вернуть $\bot$ для составного $n$): $2^{-2k}$


\section{Тест на простоту, основанный на эллиптических\\ кривых}

\textbf{Задача}: По данному (большому) числу $p$ определить, является ли $p$ простым числом и, если да, вывести доказательство (\textit{сертификат}) простоты $p$. 

Самый быстрый на сегодняшний день \underline{вероятностный} алгоритм предложен\\ S. Goldwasser,J. Killan «Primality testing using elliptic curves» в 1986.\\ С последующими улучшениями, он работает за время $\operatorname{poly}\log p$, проверка сертификата простоты: $\bigO(\log^4 p)$

\underline{Детерминированные} алгоритмы (Cohen, Lenstra «Primality testing \& Jacobi Sums» 1984) работают за \underline{квази}-полиномиальные от $\log p$ время $(\log p)^{\bigO(\log\log p)}$

Т.е. детерминированные алгоритмы пригодны для небольших чисел $p$.

\subsection{Предварительные сведения}

\begin{theorem}[О распределении порядков случайных эллиптических кривых]
\label{t1}

Пусть $p>5$ — простое. $S \subseteq \{ p+1-\lfloor\sqrt{p}\rfloor, p+1+\lfloor\sqrt{p}\rfloor \}$.\\
Пусть $A, B \leftarrow^{\$} \F_p$. Тогда $\exists\ c$ — константа $(c \in \theta(1))$, т.ч.
\[
    Pr\left[\#E_{A,B}(\F_p) \in S \right] > \frac{k}{\log p} \cdot \frac{|S|-2}{2\lfloor\sqrt{p}\rfloor + 1}
\]
где $\#E_{A,B}(\F_p)$ — число $\F_p$-рациональных точек на кривой $E_{A,B}: y = x^3 + Ax+B$

Неформальная интерпретация теоремы: число точек $E_{A,B}$ ведёт себя как случайное число из интервала $\left[p+1-\lfloor\sqrt{p}\rfloor, p+1+\lfloor\sqrt{p}\rfloor \right]$
\end{theorem}

\begin{lemma}
\label{l2}
    Пусть $n \in \Z,\ 2,3\nmid \ n;\; p>3$ — простой делитель $n$ и $4A^3 + 27B^2 \not\equiv 0\bmod p$.
    
    Для любого $x \in \Z/n\Z$ зададим $x_p := x\bmod p$ и\\ для любой точки $L = (x,y) \in E_{A,B}( \Z/n\Z)$ зададим $L_p = (x_p, y_p) \in E_{A,B}(\F_p)$, \\
    $\bigO_p = \bigO_x \in E_{A, B}(\Z/n\Z)$
    
    Тогда $\forall L, M \in E_{A,B}(\Z/n\Z)$ если $L+M$ определено, то $(L+M)_p = L_p+M_p$
\end{lemma}
\begin{proof}
    Проверить формулы сложения для $E_{A,B}(\Z/n\Z)$, см. лекцию №7.
\end{proof}

\begin{theorem}[Критерий простоты]
\label{t3}
    Пусть $n \in \Z,\; 2,3 \nmid \ n$. \\
    Пусть далее $A, B \in \Z/n\Z$ т.ч. $\gcd(4A^3, 27B^2, n) =1$ и $L \neq \bigO$ на $E_{A,B}(\Z/n\Z)$.\\
    Тогда если существует простое $q > (n^{1/4} + 1)^2$, т.ч. $qL = \bigO$, то $n$ — простое. 
\end{theorem}
\begin{proof}
    От противного: пусть $n$ — составное $\Rightarrow \exists p > 3$, т.ч. $p | n$ и $p \leq \sqrt{n}$.
    
    Заметим, $\gcd(4A^3, 27B^2, p) \neq 0 \bmod p$ (иначе мы бы получили противоречие с условием $\gcd(4A^3, 27B^2, n) = 1$.
    
    Тогда по Лемме \ref{l2}: $L_p \in E_{A,B}(\F_p)$ и $q\cdot L_p = (qL)_p = \bigO_p = \bigO \Rightarrow \operatorname{ord}(L_p)$ должен делить $q$.
    
    По Теореме Хассе, $\operatorname{ord}(L_p) \leq \#E_{A,B}(\F_p) \leq (\sqrt{p} + 1)^2 \leq (n^{1/4} + 1)^2 < q$. \\
    Это противоречие, значит, $n$ — простое.
\end{proof}

\subsection{Алгоритм: тест на простоту}

Идея: Сведём доказательство простоты $p$ к доказательству простоты $q \leq \frac{p}{2} + o(P)$, рекурсивно применим алгоритм к $q$, пока не получим достаточно малое значение $q$ — такое, что \underline{детерминированные} тесты будут эффективны.

Для заданного $p$, построим кривую $E_{A,B}$ над $p$ с точкой $L \in E_{A,B}(\F_P)$ порядка $q \approx p/2$

Условия генерации $A, B$:
\begin{enumerate}[label=\alph*)]
    \item $(4A^3 + 27B^2, p) = 1$
    \item $\#E_{A,B}(\F_p) \in \left[ p+1 - \lfloor\sqrt{p}\rfloor, p + 1 + \lfloor\sqrt{p}\rfloor \right]$\\
    Использует эффективные алгоритмы подсчёта точек на кривой, например, алгоритм Схофа
    \item $\#E_{A,B}(\F_p)$ чётно
\end{enumerate}

\begin{algorithm}[H]
	\caption{gen\_curve}
	\label{alg:GenCurveP}
    \textbf{Input:} $p$
    
    \textbf{Output:} $A,B,q$

	\begin{algorithmic}[1]
		
		\State $A, B \xleftarrow{\$} \F_p$ по условиям выше
		\State $q = \#E_{A,B}(\F_p) / 2$
		\If {$2\ |\ q$ или $3\ |\ q$} 
		    \State Вернуться к шагу 1  
	    \EndIf
	    \State Запустить вероятностный алгоритм проверки $q$ на простоту (Миллера—Рабина) на $\bigO(\log p)$ шагов (т.е. чтобы вероятность ошибки была $\sim 2^{-\log p}$).
	\end{algorithmic}
\end{algorithm}

\begin{algorithm}[H]
	\caption{find\_point}
	\label{alg:Find_point}
    \textbf{Input:} $p, q, A, B$

	\begin{algorithmic}[1]
		\State $x \xleftarrow{\$} \F_p$, что $x^3+Ax+B$ — квадрат в $\F_p$
		\State $y \xleftarrow{\$} \left\{ \pm \sqrt{x^3 + Ax+B} \right\}; L:=(x,y)$
		\If{$q \cdot L \neq \bigO$}
		    \State вернуться к шагу 1.
		\EndIf
		\State \Return L
	\end{algorithmic}
\end{algorithm}

$LB$ — число бит в числе такое, что детерминированные алгоритмы простоты эффективны для этого числа.

\begin{algorithm}[H]
	\caption{prove\_prime}
	\label{alg:Prove_prime}
    \textbf{Input:} $p$

	\begin{algorithmic}[1]
		\State $i = 0, p_0 = p$
		\While{$p_i > 2^{LB}$}
		    \State $(A_i, B_i), p_{i+1} \leftarrow \text{gen\_curve}(p)$ \label{prove_prime_2_1}
		    \State $L_i \leftarrow \text{find\_point} (p_i, p_{i+1}, A, B)$\label{prove_prime_2_1} \label{prove_prime_2_2}
		    \State $i := i+1$
		    \If{$i \geq (\log p)^{\log\log p}$ или $2\ |\ p_i$ или $3\ |\ p_i$}
		        \State Вернуться к шагу 1
		    \EndIf
		\EndWhile
		\State Проверить $p_i$ на простоту детерминированным алгоритмом (\textit{В лабе — тривиальным делением на числа до $\sqrt{p_i}$}) \label{prove_prime_3}
		\If{$p_i$ не доказано простым}
		    \State Вернуться к шагу 1
		\EndIf
		\State \Return $C = ((A_0, B_0), L_0, p_1, ..., (A_{i-1}, B_{i-1}), L_{i-1}, p_{i-1})$
	\end{algorithmic}
\end{algorithm}

\subsubsection{Корректность}
\begin{itemize}
    \item $p$ — простое. Тогда выход $C$ — сертификат: "свидетельство" простоты $p$. На шагах 3, 4 мы получаем кривую $E_{A_i, B_i}$  и точку $L_i$ порядка $p_{i+1}$, удовлетворяющие условиям Теоремы 3.
    \item $p$ — составное. Тогда получим делители $p$ на шаге \ref{prove_prime_2_2} (или раньше). $(2, 3 \nmid p)$ алгоритма find\_point(), аналогично алгоритму факторизации. 
\end{itemize}

\subsubsection{Корректность}
\underline{Alg. \ref{alg:GenCurveP}} Самый затратный шаг — подсчёт $\#E_{A,B}(\F_p)$ (условия генерации, пункт b).\\
Алгоритм Схофа: $\widetilde\bigO(\log^8 p)$. Вероятность, что $\#E_{A,B}(\F_p)$ лежит в нужном интервале — Теорема \ref{t1}

\underline{Alg. \ref{alg:Find_point}} Самые затратные шаги:

Шаг 1: $x \xleftarrow{\$} \F_p$ — кв. вычет с вероятностью $\bigO(1)$.

Шаг 4: быстрое умножение на $q$:\\
$\bigO(\log q \cdot \log^2 p) = \bigO(\log^3 p)$

\underline{Alg. \ref{alg:Find_point}} В каждой Итерации шага 2, $p_i$ уменьшается на 2, т.е. ожидаем $\bigO(\log p)$ итераций.\\
Доминирующий шаг: условия генерации для gen\_curve() — (b)\\
$\Rightarrow $ общее время работы: $\bigO(\log^9 p)$

Количество кривых $E_{A,B}$, не удовлетворяющих свойствам (a)—(c) условий генерации для gen\_curve() $= \bigO(\log^3 p)$ (эвристика)

\subsection{Проверка сертификата}

\begin{algorithm}[H]
	\caption{check\_prime}
	\label{alg:check_prime}
    \textbf{Input:} $p_0, C =((A_0, B_0), L_0, p_1, ..., (A_{i-1}, B_{i-1}), L_{i-1}, p_{i-1})$
    
    \textbf{Output:} \{Reject, Accept\}

	\begin{algorithmic}[1]
		\For{$j = 0\ ...\ i-1$}
            \State assert $(2 \nmid p_j)$ \textit{(a)}
            \State assert $(3 \nmid p_j)$ \textit{(b)}
            \State assert $(4A^3_j + 27B_j^2, 1) = 1$ \textit{(c)}
            \State assert $(P_{j+1} > (p_j^{1/4} + 1)^2)$ \textit{(d)}
            \State assert $L_j \neq \bigO$ \textit{(e)}
            \State assert $p_{j+1}L_j = \bigO$\textit{(f)}
		\EndFor
		\State \Return Accept
	\end{algorithmic}
\end{algorithm}

\subsubsection{Корректность}
Если check\_prime() возвращает Accept, $\Rightarrow p_i$ — простое $\Rightarrow p_{i-1}$ простое по Теореме \ref{t3} ($\Rightarrow\ ...\ \Rightarrow p_0$ — простое)

Условия (a),(b) проверяются на шаге 6 алгоритма \ref{alg:Prove_prime}

(c) — шаг (a) условия генерации

(d) — Теорема Хассе: $\#E_{A,B}(\F_{p_j}) \geq (\sqrt{p_j} - 1)^2 \Rightarrow$
\[
    p_{j+1} = \frac{\#E(\F_{p_j}}{2} \geq \frac{(\sqrt{p_j} - 1)^2}{2} > (p_j^{1/4} + 1)^2\; \forall p_j > 37
\]
Для столь малых $p_j$ проверка на простоту тривиальна.

(e), (f) проверяются в find\_point, шаг 3.

\subsubsection{Время работы}
Проверка каждого $p_j: \bigO(\log^3 p)$ — шаг (f) самый затратный.

Всего: $\bigO(\log p)$ различных $p_j$ в сертификате $C \Rightarrow \bigO(\log^4 p)$

\end{document}